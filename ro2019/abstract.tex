
\begin{abstract}
 Research Objects have the potential to significantly enhance the reproducibility
of scientific research.  One important way Research Objects can do this
is by encapsulating the means for re-executing the computational components
of studies, thus supporting the new form of reproducibility enabled by digital
computing---exact repeatability.  However, Research Objects also can make
scientific research more reproducible by supporting transparency, a component
of reproducibility orthogonal to re-executability.  We describe here our vision for
making Research Objects more transparent by providing means for disambiguating
claims about reproducibility generally, and computational repeatability specifically.
We show how support for science-oriented queries can enable researchers to
assess the reproducibility of Research Objects and the individual methods and results
they encapsulate.
\end{abstract}
