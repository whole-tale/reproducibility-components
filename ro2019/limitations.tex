\section{Maintaining repeatability}\label{sec-limitations}

Another way Research Objects can contribute to scientific transparency is by
	clarifying claims about computational repeatability.
Just as the overall scientific reproducibility of a study represented by a Research Object
	might be described precisely in terms of individual components required to
	satisfy particular (namespaced) definitions of \emph{reproducible} or \emph{replicable},
	additional statements could be made about the various dimensions of computational
	reproducibility in particular.

There have long been discussions about packaging ROs for transparency and
    reproducibility \cite{claerbout1992, gentleman2007, peng2011}, but less emphasis has been
    placed on \emph{describing} characteristics of the published ROs. The aim would be for researchers
    publishing their work via ROs to be fully aware of the implications of the claims they make about
    the computations represented by the RO.
Researchers discovering, evaluating, or using the Research Object for further research
	 would be able to interpret these claims unambiguously.
The implications for the possibilities of rerunning, reproducing, or exactly repeating the
	computations described in the RO under different conditions would be clear
	to all parties.

As discussed above, \emph{exact repeatability} promises increasingly to be a powerful,
	new addition to the modern researcher's repertoire of reproducibility techniques.
At the same time, there appears to be (possibly growing) confusion over what is actually
	possible in terms of computational reproducibility generally, exact repeatability
	specifically, and the conditions required to achieve them in practice.

The fundamental limitations computers impose on the exact replicability of program executions
	are well known.
At the lowest level, finite precision arithmetic, differing word sizes between processors,
	the effects of round-off errors, and the implications of choosing between different
	mathematically equivalent orderings of operations all have the potential to
	impose limits on the replicability of scientific computations across different computing environments.
Virtual machines and software containers cannot fully address issues at this level.

The fact that we can expect options such as full-processor emulation, either in software~\cite{QEMU}
	or on customizable hardware,
	to provide better guarantees of exact computational repeatability under more circumstances
	over time reveals the true crux of the problem.
What we can expect from computers in terms of reproducibility in general, and exact
	repeatability in particular, is changing quickly---and likely will continue to do so for the
	foreseeable future.
In the case of hiding hardware differences, time is on our side---or can be if we happen to save the
	information actually needed to enable exact re-execution of our analyses in the future.

In many other cases, time works against repeatability~\cite{hinsen_dealing_2019}.
A Dockerfile that today correctly produces the software environment in which computations were originally
	performed may not do so a year from now--if it builds at all.
Due to the dependencies of most scientific software on packages not
	bundled with the language compiler or runtime (with these packages typically depending on
	other packages, and so on), the chances of rebuilding or rerunning software
	equivalent to that used to produce a result in a Research Object decreases
	rapidly with time.
Fortunately, time also works for us in this dimension as well, as new ways of specifying software
	environments and archiving dependencies emerge.
But again the issue arises---are we saving the right information to enable computational repeatability
	in the future?

What Research Objects can offer here is analogous to the proposed function of mediating between competing
	and contradictory definitions of reproducibility and replicability.
Rather than trying to anticipate all future developments in the area of computational reproducibility,
	and representing computing environments, software dependencies, and machine information
	in a way that we hope will be usable by future technologies, we can take the pluralistic
	path here as well.
We can characterize the various dimensions in which computing technology currently supports---or fails to
	support---exact repeatability; then create mappings from the specific capabilities
	of existing technologies (Docker~\cite{docker2019}, Singularity~\cite{Singularity2019}, Jupyter~\cite{jupyter2019}, etc)
	and software stacks (Binder~\cite{Binder_2018}, Whole Tale~\cite{WT2019}, etc)
	onto these dimensions.
As new technologies that better (or differently) support computational reproducibility emerge or gain acceptance,
	the capabilities of these tools can be mapped as well; and the common, underlying model can be enhanced as needed.

The advantage of including such capabilities in Research Objects is that researchers could be made aware
	of the implications of the various technologies, programming environments, and specification
	standards they choose to use employ.
It is easy to imagine a current-day researcher intending to enable others to reproduce their computational
	results by sharing any custom source code or scripts in a Git repository, along with the Dockerfile
	they used to create the computing environment in which they worked.
While this is laudable, and almost certainly better than nothing, in many cases it is likely the researcher's
	expectations with regard to how these actions will ensure reproducibility will exceed
	what is actually the case.
If instead the researcher composed their study as a Research Object, they could be prompted---by whatever
	software environment they are using to create the RO---for details about their precise expectations with
	regard to reproducibility.
They may then find that a Dockerfile that does not specify the version of the base image, for example, is not sufficient
	to meet their expectations.
When faced with the current limitations of available technology they may choose to archive the Docker image itself,
	or even a virtual machine image, while still being made aware of the limitations associated with these alternative
	approaches.

Researchers evaluating an RO similarly could probe its reproducibility capabilities.  One might discover for example
	that an RO comes with a Dockerfile that currently references a non-existent base image.
Or that it depends on a software package no longer available in the Ubuntu Apt Repository.
The archived Docker image referenced in the RO might no longer be compatible with the latest version of Docker.
These are all challenging issues to discover, debug, and remediate even for experts in these technologies.
Making Research Objects \emph{transparent} with respect to their actual reproducibility capabilities would be a
	step forward for making the computational components of scientific research reproducible and
	transparent.

%%% Local Variables:
%%% mode: latex
%%% TeX-master: "main"
%%% End:
